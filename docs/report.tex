\documentclass[12pt]{article}
\usepackage{xeCJK}

\usepackage{listings}
\usepackage{xcolor}

\usepackage[a4paper, left=3.17cm, right=3.17cm, top=2.54cm, bottom=2.54cm]{geometry}
\usepackage{fancyhdr} % header and footer
\usepackage[T1]{fontenc}
\usepackage{mathptmx}
\usepackage{amsmath}
\usepackage{amsfonts}
\usepackage{chemformula}
\usepackage{cite}
\usepackage[colorlinks, linkcolor=black, anchorcolor=black, citecolor=black]{hyperref}
\usepackage{graphicx}
\usepackage{hyperref}

\setCJKmainfont{Songti SC}
\setCJKsansfont{Songti SC}
\setCJKmonofont{Songti SC}

\lstset{ %
  language=C++,
  backgroundcolor=\color{black!5},
  basicstyle=\footnotesize,
}
% show paragraphs in table of contentes
\setcounter{tocdepth}{4}
\setcounter{secnumdepth}{4}

\setlength{\parskip}{0.5em}
\title{C语言词法分析程序的设计与实现}
\author{\textup{吴清柳}}
\begin{document}
\begin{titlepage}
    \newcommand{\HRule}{\rule{\linewidth}{0.5mm}}
    \includegraphics[width=8cm]{title/logo_bupt.png}\\[1cm]
    \center
    \quad\\[1.5cm]
    \textsl{\Large 北京邮电大学}\\[0.5cm]
    \textsl{\large  计算机学院}\\[0.5cm]
    \makeatletter
    \HRule \\[0.4cm]
    {\huge \bfseries \@title}\\[0.4cm]
    \HRule \\[1.5cm]
    % \begin{minipage}{0.4\textwidth}
    %     \begin{flushleft} \large
    %         \emph{Author:}\\
    %         \@author
    %     \end{flushleft}
    % \end{minipage}
    % ~
    % \begin{minipage}{0.4\textwidth}
    %     \begin{flushright} \large
    %         \emph{Supervisor:} \\
    %         \textup{ 王雅文}
    %     \end{flushright}
    % \end{minipage}\\[3cm]
    \makeatother
    {\large (C++版和FLEX版均实现)}\\[0.5cm]
    {\large 姓名: 吴清柳}\\[0.5cm]
    {\large 学号: 2020211597}\\[0.5cm]
    {\large 班级: 2020211323}\\[0.5cm]
    {\large 指导老师: 王雅文}\\[0.5cm]
    {\large \emph{课程名称: 编译原理}}\\[0.5cm]
    {\large \today}\\[2cm]
    \vfill
\end{titlepage}

\tableofcontents
\newpage

% set page stype to fancy then decorate it
\pagestyle{fancy}
\fancyhead{} % clear all header fields
\fancyhead[LE, RO]{吴清柳 2020211597}
\fancyhead[LO, RE]{C语言词法分析程序的设计与实现}
\fancyfoot{} % clear all footer fileds
\fancyfoot[CE, CO]{\thepage}

\section{题目及要求}
\begin{enumerate}
  \item 设计一个C语言词法分析程序;
  \item 可以识别出用C语言编写的源程序中的每个单词符号, 运算符, 数字等,
  并以<记号, 属性>的形式 输出每个单词符号;
  \item 可以识别并跳过源程序中的注释;
  \item 可以统计源程序中的语句行数, 各类单词个数, 字符总数, 并输出统计结果;
  \item 可以识别C语言程序源代码中存在的词法错误, 并报告错误出现的位置;
  \item 对源程序中出现的错误进行适当的恢复, 让词法分析可以继续进行;
  \item 对源程序进行一次扫描, 即可检查并报告源程序中存在的所有词法错误, 并输出
  源程序中所有的记号.
\end{enumerate}

\section{实验设备}
Ubuntu 20.04.5 LTS on Windows 10 x86\_64, \\[0.5cm]
macOS 12.6 21G115 arm64, \\[0.5cm]
Neovim v0.7.2, \\[0.5cm]
clang-1400.0.29.102, \\[0.5cm]
flex 2.6.4 \\[0.5cm]

\section{程序设计说明}
\input{content/3.tex}
\section{实验流程图}
\begin{figure}
  \begin{center}
    \includegraphics[width=0.8\textwidth]{figures/Page1.jpg}
  \end{center}
  \caption{词法分析程序的状态转移图 1}
  \label{fig:StateFig1}
\end{figure}

\begin{figure}
  \begin{center}
    \includegraphics[width=0.8\textwidth]{figures/Page2.jpg}
  \end{center}
  \caption{词法分析程序的状态转移图 2}
  \label{fig:StateFig2}
\end{figure}

\begin{figure}
  \begin{center}
    \includegraphics[width=0.8\textwidth]{figures/Page3.jpg}
  \end{center}
  \caption{词法分析程序的状态转移图 3}
  \label{fig:StateFig3}
\end{figure}
\newpage

\section{实验程序}
出于模块化, 解耦合, 可扩展性和可读性考虑, 将词法分析程序划分为\textbf{符号和符号表}
与\textbf{词法分析处理和输入输出}两个模块, 分别在 Symbol.h, Symbol.cpp, Lex.h和Lex.cpp
四个源代码文件中进行实现, 并使用CMake作为构建工具.

其中, Symbol.h和Symbol.cpp对记号(Symbol)及记号表(SymbolList)的类进行成员和方法定义和实现,
 Lex.h和Lex.cpp对词法分析处理, 输入和输出类Lex进行成员和方法定义和实现.

main.cpp承担着作为词法分析程序入口的功能.

\subsection{main.cpp}
词法分析程序入口
\lstinputlisting{../src/main.cpp}
\subsection{Symbol.h}
对记号(Symbol)及记号表(SymbolList)的类成员和方法定义;
\lstinputlisting{../src/Symbol.h}
\subsection{Symbol.cpp}
对记号(Symbol)及记号表(SymbolList)的类成员和方法实现;
\lstinputlisting{../src/Symbol.cpp}
\subsection{Lex.h}
词法分析处理和输出类Lex的成员和方法定义;
\lstinputlisting{../src/Lex.h}
\subsection{Lex.cpp}
词法分析处理和输出类Lex的成员和方法实现;
\lstinputlisting{../src/Lex.cpp}

\section{实验输入(测试程序)}
\subsection{测试样例1}
\lstinputlisting{../build/sin_1}
\subsection{测试样例2}
\lstinputlisting{../build/sin_2}

\section{实验运行结果及分析说明}
\input{content/7.tex}
\section{心得体会}
\input{content/8.tex}
\section{附录}
代码仓库: \href{https://github.com/Micuks/Lexical\_Analysis}{Micuks/Lexical\_Analysis}

% \bibliographystyle{ieeetrans}
% \bibliography{Assignment_Ref}

\end{document}
