\begin{enumerate}
\item 词法分析程序的作用:
  \begin{enumerate}
  \item 扫描源程序字符流;
  \item 按照源语言的词法规则识别出各类单词符号;
  \item 产生用于语法分析的记号序列;
  \item 词法检查;
  \item 创建符号表, 将识别出来的标识符放入符号表中;
  \item 跳过源程序中的注释和空格等, 把错误信息和源代码联系起来;
  \end{enumerate}
\item 源程序代词类别:
  \begin{enumerate}
    \item 关键字;
    \item 用户定义变量标识符;
    \item 数字, 字符和字符串常量;
    \item 运算符;
    \item 分隔符;
  \end{enumerate}
\item 设计思路:
  利用有限状态自动机模型, 将整个源代码分析的过程转化为不同状态之间的转移, 在画好
  状态转移图之后, 借用C++的switch语句或if/else语句将状态转移图描述出来. 此外, 
  实现好读取源代码, 缓冲区, 以及输出分析结果, 和将常量和变量名插入到符号表的功能.
\item 伪代码描述:
\begin{lstlisting}
// Initializing...
while(End of source file not reached) {
  ch = getchar();
  switch(ch) {
    determine state based on the input char
    case 0:
    ...
  }
  switch(State) {
    // Limited state machine process
    case 0:
      // state 0 process
    case 1:
      // state 1 process
    ...
  }
}
\end{lstlisting}
\end{enumerate}
