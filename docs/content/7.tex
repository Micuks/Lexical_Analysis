\subsection{输出格式概述}
\begin{itemize}
  \item 在Statistic information之前的是输入词法分析程序的源代码文件中解析到的符号, 格式为
L[行号]: <[记号], [属性]>.
  \item Statistic information(统计信息)有如下格式:
  \begin{itemize}
    \item 第一行为行数, 字符数和符号数统计, 格式为[行数] lines, [字符数] characters,
    [符号数] symbols;
    \item 之后跟[符号数]行, 每行是一个符号及其出现次数, 格式为<[记号], [属性]> appeared
     [出现次数] times;
  \end{itemize}
\end{itemize}

\subsection{测试样例1的输出}
测试样例1为较简短的样例, 用于进行输入输出功能和状态转移基本功能的测试;
\subsubsection{C++版词法分析程序测试}
\lstinputlisting{code_lists/sin_1_output}
\subsubsection{FLEX版词法分析程序测试}
\lstinputlisting{code_lists/sin_1_flex}
\subsection{测试样例2的输出}
测试样例2为较长, 测试内容较全面的样例, 用于对词法分析程序实现的各项功能进行测试.
\subsubsection{C++版词法分析程序测试}
\lstinputlisting{code_lists/sin_2_output}
\subsubsection{FLEX版词法分析程序测试}
\lstinputlisting{code_lists/sin_2_flex}
\subsection{分析和总结}
通过C++版和FLEX版分析结果的对照, 说明该C语言词法分析程序有以下功能:
\begin{enumerate}
  \item 可以识别出用C语言编写的源程序中的每个单词符号, 运算符, 数字等,
  并以<记号, 属性>的形式 输出每个单词符号, 包括转义字符;
  \item 可以识别并跳过源程序中的注释;
  \item 可以统计源程序中的语句行数, 各类单词个数, 字符总数, 并输出统计结果;
  \item 可以识别C语言程序源代码中存在的词法错误, 并报告错误出现的位置;
  \item 对源程序中出现的错误进行适当的恢复, 让词法分析可以继续进行;
  \item 对源程序进行一次扫描, 即可检查并报告源程序中存在的所有词法错误, 并输出
  源程序中所有的记号.
\end{enumerate}
